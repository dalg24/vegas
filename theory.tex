\section*{THEORY}
Method of Manufactured Solutions to verify convergence to exact solutions


\subsection*{Governing equations}
2D nonlinear transient problem involving neutronics and heat conduction

%%%%%%%%%%%%%%%%%%%%%%%%%%%%%%%%%%%%%%%%%%%%%%%%%%%%%%%%%%%%%%%%%%%%%%%%%%%%%%%%
%%%%  HEAT CONDUCTION
%\subsection*{$\rightarrow$Heat conduction}
Let us solve the following equation for the temperature $T(\bx, t)$
\begin{align}
  \rho c_p \frac{\partial T}{\partial t} 
  - \nabla\cdot(k(T) \nabla T) 
  - \kappa \Sigma_{f} \phi(\bx,t)&  \nonumber \\
  - Q_T(\bx,t) &= 0 \label{heatcond_eq}
\end{align}
on a rectangular domain $\Omega$ with a homogeneous (zero) Dirichlet boundary condition $T(\bx,t) = 0$ on $\partial\Omega$, where the thermal conductivity $k$ depends on the temperature $T$
\begin{align}
  k(T) = k(T_{ref}) + \left. \frac{\partial k}{\partial T}\right|_{ref} \left( T - T_{ref} \right)
\end{align}

$Q_T$ is the extraneaous temperature source and will be determined by the choice of the exact solution in the MMS procedure.  This is a function of space and time a priori.

The weak formulation of the time discretized (implicit Euler) heat conduction problem is derived in the standard way, by first multiplying the equation \eqref{heatcond_eq} with the test functions $u_{i}$ and then integrating over the domain $\Omega$
\begin{align}
  \int_{\Omega} \rho c_p \frac{T^{n+1}-T{^n}}{\tau} u_i d\bx 
  - \int_{\partial\Omega} k(T^{n+1}) \nabla T^{n+1} u_i \cdot \bs{n} d\bx&  \nonumber \\
  + \int_{\Omega} k(T^{n+1}) \nabla T^{n+1} \cdot \nabla u_i d\bx
  - \int_{\Omega} \kappa_1 \Sigma_{f1} \phi_1^{n+1} u_i d\bx& \nonumber \\
  - \int_{\Omega} Q_T^{n+1} u_i d\bx = 0& \label{heatcond_wf}
\end{align}
(constant time step $\tau$)

%%%%%%%%%%%%%%%%%%%%%%%%%%%%%%%%%%%%%%%%%%%%%%%%%%%%%%%%%%%%%%%%%%%%%%%%%%%%%%%%
%%%%  NEUTRONICS
%\subsection*{$\rightarrow$Neutronics}
The neutron flux $\phi(\bx,t)$ is described by means of a simple one-group neutron diffusion equation
\begin{align}
  \frac{1}{v} \frac{\partial\phi}{\partial t} 
  - \nabla\cdot(D \nabla \phi) 
  + \Sigma_{a}(T) \phi 
  - \nu \Sigma_{f} \phi_1& \nonumber \\
  - Q(\bx,t) &= 0 \label{neutron_eq}
\end{align}

Doppler feedback is accounted in the model through the removal cross section $\Sigma_{a}$, which is affected by temperature variations.
\begin{align}
  \Sigma_{a}(T) = \Sigma_{a}(T_{ref}) 
  + \left. \frac{\partial \Sigma_{a}}{\partial \sqrt{T}}\right|_{ref} 
    \left( \sqrt{T} - \sqrt{T_{ref}} \right)
\end{align}

The forcing function $Q$ in the equation \eqref{neutron_eq} will be determined later when choosing the exact solutions, and substituting these into the governing equations.

Finally, with the homogeneous zero Dirichlet condition, the weak formulation of the neutronics problem reads
\begin{align}
  \int_{\Omega} \frac{1}{v} \frac{\phi^{n+1}-\phi{^n}}{\tau} u_i d\bx 
  - \int_{\partial\Omega} D \nabla \phi^{n+1} u_i \cdot \bs{n} d\bx& \nonumber \\
  + \int_{\Omega} D \nabla \phi^{n+1} \cdot \nabla u_i d\bx 
  + \int_{\Omega} \Sigma_{a}(T^{n+1}) \phi^{n+1} u_i d\bx& \nonumber \\
  - \int_{\Omega} \nu \Sigma_{f} \phi^{n+1} u_i d\bx 
  - \int_{\Omega} Q^{n+1} u_i d\bx = 0&
\end{align}



\subsection*{Newton's method}
For fixed time level $n$ we will solve the time independent nonlinear problem for the temperature $T^{n}$ and the neutron flux $\phi^{n}$ by means of the Newton's Method.

Let $\bs{Y}^n = (y_1^n,y_2^n,...,y_N^n)^T$ be the vector of unknown coefficients so that we have

At each Newton iteration $k$ we solve
\begin{align}
  \bs{Y}_{k+1}^{n} = \bs{Y}_{k}^{n} 
  - \bs{J}^{-1}(\bs{Y}_{k}^{n}) \bs{F}(\bs{Y}_{k}^{n})
\end{align}
or
\begin{align}
  \bs{J}(\bs{Y}_{k}^{n}) \delta\bs{Y}_{k}^{n} = - \bs{F}(\bs{Y}_{k}^{n}) \label{newton}
\end{align}
where $\bs{Y}_{k+1}^{n} = \bs{Y}_{k}^{n} + \delta\bs{Y}_{k}^{n}$, $n$ and $k$ being indices for time and Newton iterations respectively.

All we need to do is to provide the code with the Jacobian matrix $\bs{J}^{n}=\bs{J}(\bs{Y}^{n})=D\bs{F}^{n}/D\bs{Y}^{n}$
and the residual vector $\bs{F}^{n}=\bs{F}(\bs{Y}^n)=\bs{0}$.

\subsection*{Manufactured Solution}
Choice of
\begin{align}
  T(x,y,t) = 
  C_T \left(1+\tanh(r_T t)\right)
  \sin\left(\frac{\pi x}{L_X}\right)
  \sin\left(\frac{\pi y}{L_Y}\right)
\end{align}
and
\begin{align}
  \phi(x,y,t) = 
  C_F \left(1+\exp(r_F t)\right)
  \sin\left(\frac{\pi x}{L_X}\right)
  \sin\left(\frac{\pi y}{L_Y}\right)
  \frac{x}{L_X}
  \frac{y}{L_Y}
\end{align}

